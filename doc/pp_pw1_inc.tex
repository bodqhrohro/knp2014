\begin{tikzpicture}
\path[
 mindmap,
 scale=0.8,
 concept color=white,
 level 1 concept/.append style={sibling angle=120},
 level 2 concept/.append style={sibling angle=60},
 level 3 concept/.append style={sibling angle=30},
 sys/.style={concept, font=\footnotesize}
]
node [sys] {Система обліку платних курсів\\із підтримкою самостійного запису слухачів через мережу Інтернет} [clockwise from=210]
 child {
  node {Вхідний потік} [clockwise from=330]
   child { node {Хто?} [clockwise from=340]
    child { node {Оператори} }
    child { node {Адміністратор} }
    child { node {Слухачі}
     child { node {Студенти} }
     child { node {Сторонні} }
    }
   }
   child { node {Що?} [clockwise from=320]
    child { node {Дані слухачів} }
    child { node {Курси} }
    child { node {Дані викладачів} }
    child { node {Коефіціенти} }
    child { node {Розклад курсів} }
   }
   child { node {Де?} [clockwise from=200]
    child { node {Web-сайт} }
    child { node {Кафедра СПЗ} }
   }
   child { node {Коли?}  [clockwise from=180]
    child { node {Початок чверті} }
    child { node {Довільно} }
   }
   child { node {Як?} [clockwise from=150]
    child { node {Ручне введення} }
    child { node {Імпорт з системи електронного деканату} }
   }
 } child {
  node {Внутрішній потік} [clockwise from=210]
   child { node {Хто?} }
   child { node {Що?} 
    child { node {Збереження даних} }
    child { node {Контроль цілісності} }
    child { node {Розрахунок вартості курсу} }
   }
   child { node {Де?} [clockwise from=105]
    child { node {Web-сервер} }
    child { node {СКБД} }
   }
   child { node {Коли?} [clockwise from=60]
    child { node {Постійно} }
   }
   child { node {Як?} [clockwise from=30]
    child { node {Формула} }
   }
 } child {
  node {Вихідний потік} [clockwise from=90]
   child { node {Хто?} [clockwise from=120]
    child { node {Викладачі} }
    child { node {Оператори} }
    child { node {Адміністратор} }
    child { node {Слухачі} [clockwise from=10]
     child { node {Студенти} }
     child { node {Сторонні} }
    }
   }
   child { node {Що?} [clockwise from=35]
    child { node {Сповіщення} }
    child { node {Звіти} 
     child { node {За період} }
     child { node {За курсом} }
    }
    child { node {Розклад} }
   }
   child { node {Де?} [clockwise from=30]
    child { node {Web-сайт} }
    child { node {Кафедра СПЗ} }
    child { node {Бухгалтерія} }
   }
   child { node {Коли?} [clockwise from=340]
    child { node {Кінець семестру} }
    child { node {Під час курсів} }
   }
   child { node {Як?} [clockwise from=310]
    child { node {Звіти} 
     child { node {У файл} }
     child { node {Друк} }
    }
    child { node {Сповіщення} [clockwise from=280]
     child { node {На e-mail} }
     child { node {На сайті} }
    }
   }
 };
\end{tikzpicture}
\newpage

\section*{Проблеми}

\begin{itemize}
 \item Відсутність актуальної інформації про наявні курси в мережі Інтернет.
 \item Необхідність відвідувати кафедру СПЗ для реєстрації на курс, черги, ручне введення контактних даних через операторів.
 \item Зберігання інформації в текстових документах та ручне формування звітів.
 \item Ручне внесення даних у бланк договору.
 \item Слухачі забувають розклад курсів, а викладачі шукають вільну аудиторію для проведення курсів.
\end{itemize}

\section*{Аналогічні системи}

\begin{itemize}
 \item 1C: Бухгалтерія
 \item Парус
 \item 1C: Бітрікс
\end{itemize}

\newpage

\section*{Діаграма варіантів використання}

\scalebox{0.4}{\input{pp_pw1_uc.tex}}

\section*{Опис варіантів використання}

\newcommand{\rowspan}[3]{
 \multirow{#1}{*}{#2} #3
}
\newcommand{\usecase}[9]{
 \begin{longtable}{|p{4,5cm}|p{12cm}|}
 \caption{#1} \\
 \hline Действующие лица & #2 \\
 \hline Область действия & #3 \\
 \hline Уровень & #4 \\
 \hline Участники и интересы & #5 \\
 \hline Предусловия & #6 \\
 \hline Гарантии успеха & #7 \\
 \hline #8
 \hline #9
 \hline
 \end{longtable}
}
\usecase{Авторизация}{Ассистент, Администратор}{Система}{Цель пользователя}{Действующее лицо авторизуется в системе}{Ключ сессии отсутствует}{Вход в систему, загрузка основного пользовательского интерфейса с соответствующими роли полномочиями.}{
\rowspan{2}{Основной сценарий}{& 1. Действующее лицо вводит свои логин и пароль.\\
\cline{2-2} & 2. Система подтверждает корректность пары логин-пароль и отображает основной интерфейс.\\ }
}{
\rowspan{3}{Расширения}{& 2.А. Введённые данные неверны.\\
\cline{2-2} & 2.А.1. Сообщение об ошибке.\\
\cline{2-2} & 2.А.2. Возвращение к пункту 1.\\ }
}
\usecase{Размещение заявки на добавление слушателя}{Гость}{Система}{Цель пользователя}{Гость создаёт заявку на добавление слушателя}{Гость принял решение самостоятельно записаться на курс}{Заявка успешно размещена.}{
\rowspan{3}{Основной сценарий}{& 1. Гость заходит на сайт кафедры.\\
\cline{2-2} & 2. Гость заполняет форму регистрации нового слушателя\\
\cline{2-2} & 3. Система подтверждает корректность заявки и сохраняет её.\\ }
}{
\rowspan{2}{Расширения}{& 3.А. Данные введены некорректно либо такой слушатель уже существует и уже записан на этот курс.\\
\cline{2-2} & 3.А.1. Сообщение об ошибке.\\
\cline{2-2} & 3.А.2. Возвращение к пункту 1\\ }
}
\usecase{Подтверждение/отклонение заявки}{Ассистент, Администратор}{Система}{Цель пользователя}{Действущее лицо решает, подтвердить поступившую заявку или нет}{Получена заявка}{Заявка исчезла из списка заявок и преобразовалась в полноценную запись}{
Основной сценарий & 1. Действующее лицо нажимает напротив текущей заявки кнопку подтверждения либо отклонения, система соответственно трансформирует заявку в запись или удаляет её. \\
}{
Расширения & ---\\
}
\usecase{Добавление слушателя}{Ассистент, Администратор}{Система}{Цель пользователя}{Действующее лицо добавляет в систему нового слушателя}{---}{Пользователь успешно добавлен и появился в списке слушателей.}{Основной сценарий & 1. Действующее лицо заполняет форму регистрации нового слушателя.\\
}{
\rowspan{3}{Расширения}{& 1.А. Данные введены некорректно либо такой слушатель уже существует.\\
\cline{2-2} & 1.А.1. Сообщение об ошибке.\\
\cline{2-2} & 1.А.2. Возвращение к пункту 1\\ }
}
\usecase{Редактирование информации о слушателе}{Ассистент, Администратор}{Система}{Цель пользователя}{Действующее лицо редактирует данные о конкретном слушателе}{Наличие слушателя в списке слушателей}{Информация о слушателе в списке слушателей соответствует введённой}{Основной сценарий & 1. Действующее лицо нажимает напротив имени пользователя иконку редактирования, исправляет данные в форме и отправляет её. \\
}{
\rowspan{3}{Расширения}{& 1.А. Данные введены некорректно либо уже существует другой пользователь с таким же ФИО.\\
\cline{2-2} & 1.А.1. Сообщение об ошибке.\\
\cline{2-2} & 1.А.2. Возвращение к пункту 1\\ }
}
\usecase{Удаление слушателя}{Ассистент, Администратор}{Система}{Цель пользователя}{Действующее лицо удаляет данные о конкретном слушателе}{Наличие слушателя в списке слушателей}{Выдано сообщение об успешном завершении операции, слушатель исчез из списка слушателей}{Основной сценарий & 1. Действующее лицо нажимает напротив имени пользователя иконку удаления.\\
}{Расширения & ---\\}
\usecase{Добавление курса}{Ассистент, Администратор}{Система}{Цель пользователя}{Действующее лицо добавляет в систему новый курс}{---}{Курс успешно добавлен и появился в списке курсов.}{Основной сценарий & 1.    Действующее лицо заполняет форму регистрации нового курса.\\
}{
\rowspan{3}{Расширения}{& 1.А. Данные введены некорректно либо курс с таким названием уже существует.\\
\cline{2-2} & 1.А.1. Сообщение об ошибке.\\
\cline{2-2} & 1.А.2. Возвращение к пункту 1\\ }
}
\usecase{Редактирование информации о курсе}{Ассистент, Администратор}{Система}{Цель пользователя}{Действующее лицо редактирует данные о конкретном курсе}{Наличие курса в списке курсов}{Информация о курсе в списке курсов соответствует введённой}{Основной сценарий & 1. Действующее лицо нажимает напротив названия курса иконку редактирования, исправляет данные в форме и отправляет её.\\
}{
\rowspan{3}{Расширения}{& 1.А. Данные введены некорректно либо уже существует другой курс с таким же названием.\\
\cline{2-2} & 1.А.1. Сообщение об ошибке.\\
\cline{2-2} & 1.А.2. Возвращение к пункту 1\\ }
}
\usecase{Удаление курса}{Ассистент, Администратор}{Система}{Цель пользователя}{Действующее лицо удаляет данные о конкретном курсе}{Наличие курса в списке курсов}{Выдано сообщение об успешном завершении операции, курс исчез из списка курсов}{Основной сценарий & 1. Действующее лицо нажимает напротив названия курса иконку удаления.\\}{Расширения & ---\\}
\usecase{Закрытие набора слушателей}{Ассистент, Администратор}{Система}{Цель пользователя}{Действующее лицо блокирует возможность дальнейшего добавления слушателей на курс}{Наличие курса в списке курсов, набор на курс ещё не закончен.}{Выдано сообщение об успешном завершении операции, курс отмечен как закрытый.}{Основной сценарий & 1. Действующее лицо нажимает напротив названия курса иконку закрытия набора.\\}{Расширения & ---\\}
\usecase{Добавление преподавателя}{Администратор}{Система}{Цель пользователя}{Администратор добавляет в систему нового преподавателя}{---}{Преподаватель успешно добавлен и появился в списке преподавателей.}{Основной сценарий & 1. Администратор заполняет форму регистрации нового преподавателя.\\
}{
\rowspan{3}{Расширения}{& 1.А. Данные введены некорректно либо преподаватель с такими ФИО уже существует.\\
\cline{2-2} & 1.А.1. Сообщение об ошибке.\\
\cline{2-2} & 1.А.2. Возвращение к пункту 1\\ }
}
\usecase{Редактирование информации о преподавателе}{Ассистент, Администратор}{Система}{Цель пользователя}{Действующее лицо редактирует данные о конкретном преподавателе (администратор – о любом преподавателе, преподаватель – о себе)}{Наличие преподавателя в списке преподавателей}{Информация о преподавателе в списке преподавателей соответствует введённой}{Основной сценарий & 1. Действующее лицо нажимает напротив ФИО преподавателя иконку редактирования, исправляет данные в форме и отправляет её.\\
}{
\rowspan{3}{Расширения}{& 1.А. Данные введены некорректно либо уже существует другой преподаватель с таким же названием.\\
\cline{2-2} & 1.А.1. Сообщение об ошибке.\\
\cline{2-2} & 1.А.2. Возвращение к пункту 1\\ }
}
\usecase{Удаление преподавателя}{Администратор}{Система}{Цель пользователя}{Администратор удаляет данные о конкретном преподавателе}{Наличие преподавателя в списке преподавателей}{Выдано сообщение об успешном завершении операции, преподаватель исчез из списка преподавателей}{Основной сценарий & 1. Действующее лицо нажимает напротив ФИО преподавателя иконку удаления.\\}{Расширения & ---\\}
\usecase{Регистрация}{Ассистент, Администратор}{Система}{Цель пользователя}{Преподаватель регистрирует аккаунт в системе, чтобы получить доступ к работе с ней, администратор подтверждает или отклоняет регистрацию аккаунта}{Наличие преподавателя в списке преподавателей, преподаватель ещё не зарегистрирован.}{Преподаватель может войти в систему либо получает уведомление о том, что его заявка отклонена.}{
\rowspan{2}{Основной сценарий}{& 1. Преподаватель выбирает себя из списка незарегистрированных преподавателей, заполняет форму заявки на регистрацию и отправляет её.\\
\cline{2-2} & 2. Администратор проверяет заявку на регистрацию и подтверждает либо отклоняет её.\\ }
}{Расширения & ---\\}
\usecase{Запись слушателя на курс}{Ассистент, Администратор}{Система}{Цель пользователя}{Действующее лицо записывает определённого слушателя на определённый курс}{Набор на курс не закрыт}{Пользователь появился в составе курса}{Основной сценарий & 1. Действующее лицо открывает перечень слушателей определённой группы и добавляет в неё нового слушателя из списка всех слушателей.\\}{Расширения & ---\\}
\usecase{Удаление слушателя с курса}{Ассистент, Администратор}{Система}{Цель пользователя}{Действующее лицо удаляет определённого слушателя с определённого курса}{Слушатель уже записан на курс}{Пользователь исчез из состава курса}{Основной сценарий & 1. Действующее лицо открывает перечень слушателей определённой группы и нажимает напротив ФИО нужного слушателя иконку исключения.\\}{Расширения & ---\\}
\usecase{Внесение оплаты}{Администратор}{Система}{Цель пользователя}{Администратор вносит оплату определённого слушателя за определённый курс}{Слушатель уже записан на курс}{Сумма, внесённая слушателем за курс, изменилась на соответствующую величину}{Основной сценарий & 1. Администратор открывает перечень слушателей определённой группы, нажимает напротив ФИО нужного слушателя иконку добавления оплаты, вводит сумму оплаты и отправляет её.\\}{Расширения & ---\\}
\usecase{Размещение заявки на запись на курс}{Гость}{Система}{Цель пользователя}{Действующее лицо размещает заявку на запись определённого слушателя на определённый курс}{Слушатель ещё не записан на этот курс}{Заявка успешно размещена}{Основной сценарий & 1. Действующее лицо выбирает группу из списка всех групп и добавляет в неё нового слушателя из списка всех слушателей.\\}{Расширения & ---\\}
