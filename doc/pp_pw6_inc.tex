\subsection{Функціональне тестування}
\tablabel{Тестові випадки для варіантів використання}\\
\newcounter{tcnt}\def\tcn{\addtocounter{tcnt}{1} \thetcnt}
\tablefirsthead{}
\tablelasttail{}
\tablehead{\hline \multicolumn{4}{|l|}{Продовження таблиці \thesection .\thetabcount} \\}
\tabletail{\hline}
\begin{supertabular}{|p{13mm}|p{65mm}|p{45mm}|p{3cm}|}
\hline
\textbf{Test Case №} &
\textbf{Дія} &
\textbf{Очікуваний результат} &
\textbf{Результат тестування (успішний (passed) / неуспішний (failed))} \\
\hline \tcn
& Перейти на сторінку реєстрації
& Відкрилася сторінка реєстрації
& 
\\ \hline \tcn
& Ввести логін test111 та двічі --- пароль test111
& Повідомлення про успішну реєстрацію
& 
\\ \hline \tcn
& Ввести логін test222 та двічі --- пароль test222
& Повідомлення про успішну реєстрацію
& 
\\ \hline \tcn
& Перейти на сторінку входу
& Відкрилася сторінка входу
& 
\\ \hline \tcn
& Ввести логін asdf та пароль asdf
& Система відхилює авторизацію
& 
\\ \hline \tcn
& Ввести логін test111 та пароль test111
& Система відхилює авторизацію
& 
\\ \hline \tcn
& Ввести логін asdf та пароль adsf
& Відкрився головний інтерфейс системи (рядок меню)
& 
\\ \hline \tcn
& Відкрити таблицю викладачів
& З'явилася таблиця
& 
\\ \hline \tcn
& Створити викладача ("Ппшш" "Шррр" "Тссс" "+3820" "ba@b.c"), зберегти зміни та оновити список
& Викладач присутній в списку та дані не пошкоджено
& 
\\ \hline \tcn
& Змінити ім'я щойно створеного викладача на "Шссс", зберегти зміни та оновити список
& Ім'я викладача змінилося, дані не пошкоджено
& 
\\ \hline \tcn
& Відкрити таблицю курсів
& З'явилася таблиця
& 
\\ \hline \tcn
& Створити курс ("" "Шррр" "Ппшш Ш. Т." "29" "Йде набір") та зберегти зміни
& Система вимагає ввести назву курсу
& 
\\ \hline \tcn
& Створити курс ("Ппшш" "Шррр" "Ппшш Ш. Т." "Ппшш Ш. Т." "29" "Йде набір"), зберегти зміни та оновити список
& Курс присутній в списку та дані не пошкоджено
& 
\\ \hline \tcn
& Змінити назву щойно створеного курсу на "Ппрр", зберегти зміни та оновити список
& Назва курсу змінилася, дані не пошкоджено
& 
\\ \hline \tcn
& Відкрити для щойно створеного курсу форму додання заняття
& Відкрилася форма
& 
\\ \hline \tcn
& Створити для щойно створеного курсу заняття ("01.04.2016" "13:30" "Лекція"), зберегти зміни та оновити список
& Заняття присутнє в списку та дані не пошкоджено
& Failed
\\ \hline \tcn
& Змінити час щойно створеного заняття на "13:45", зберегти зміни та оновити список
& Час змінився, дані не пошкоджено
& Failed
\\ \hline \tcn
& Створити для щойно створеного курсу заняття ("02.04.2016" "13:30" "Практика"), зберегти зміни та оновити список
& Заняття присутнє в списку та дані не пошкоджено
& Failed
\\ \hline \tcn
& Видалити заняття за дату 01.04.2016 та оновити список
& Заняття відсутнє у списку
& Failed
\\ \hline \tcn
& Встановити для заняття, що залишилося, тип "Лекція", зберегти зміни та оновити список
& Тип змінився, дані не пошкоджено
& Failed
\\ \hline \tcn
& Відкрити таблицю слухачів
& З'явилася таблиця
& 
\\ \hline \tcn
& Створити слухача ("Шшпп" "Ршшш" "Сттт" "АЯ-313" "+3289" "db@d.b" ""), зберегти зміни та оновити список
& Слухач присутній в списку та дані не пошкоджено
& 
\\ \hline \tcn
& Змінити ім'я щойно створеного слухача на "Ртдм", зберегти зміни та оновити список
& Ім'я слухача змінилося, дані не пошкоджено
& 
\\ \hline \tcn
& Розгорнути для щойно створеного слухача підтаблицю запису на курси
& Розгорнулася підтаблиця
& 
\\ \hline \tcn
& Записати слухача на курс "Ппрр"
& Курс з'явився в підтаблиці
& 
\\ \hline \tcn
& Перейти в таблицю "Курси"
& Таблиця перекрила інші таблиці
& 
\\ \hline \tcn
& Розгорнути для курсу "Ппрр" підтаблицю слухачів
& Підтаблиця розгорнулася і містить слухача "Шшпп Р. С."
& 
\\ \hline \tcn
& Встановити слухачеві "Шшпп Р. С." оплату "29", зберегти зміни та оновити список
& Оплата 29 грн., боргу немає
& 
\\ \hline \tcn
& Відкрити форму створення звіту за період
& Відкрилася форма
& 
\\ \hline \tcn
& Обрати дати 29.02.2016 та 28.02.2016 і створити звіт
& Система не приймає дати
& 
\\ \hline \tcn
& Обрати дати 29.02.2037 та 01.03.2037 і створити звіт
& Система не приймає дати
& 
\\ \hline \tcn
& Обрати дати 01.04.2016 та 02.04.2016 і створити звіт
& Відкривається або завантажується PDF-файл, що містить дані, відповідні даним у таблицях інтерфейсу
& 
\\ \hline \tcn
& Видалити заняття для курсу "Ппрр"
& Курс "Ппрр" не містить занять
& Failed
\\ \hline \tcn
& Видалити курс "Ппрр", зберегти зміни та оновити список
& Курс відсутній у списку
& 
\\ \hline \tcn
& Перейти в таблицю "Викладачі", видалити викладача Ппшш Шссс Тссс, зберегти зміни та оновити список
& Викладач відсутній у списку
& 
\\ \hline \tcn
& Створити у зовнішній системі три заявки, при цьому третю --- на ПІБ "Шшпу Ртдм Стут", та зачекати 20 секунд
& Лічильник сповіщень збільшився на 3
& 
\\ \hline \tcn
& Відкрити сповіщення
& Відображається панель зі сповіщеннями
& 
\\ \hline \tcn
& Підтвердити реєстрацію облікового запису test111
& Сповіщення зникло зі списку
& 
\\ \hline \tcn
& Відхилити реєстрацію облікового запису test222
& Сповіщення зникло зі списку
& 
\\ \hline \tcn
& Підтвердити першу щойно додану заявку та перейти до таблиці "Слухачі"
& Сповіщення зникло зі списку сповіщень, слухач із заявки з'явився в таблиці
& 
\\ \hline \tcn
& Відхилити другу щойно додану заявку та оновити таблицю "Слухачі"
& Сповіщення зникло зі списку сповіщень та не з'явилося в таблиці
& 
\\ \hline \tcn
& Злити третю щойно додану заявку з запропонованим "Шшпп Ртдм Сттт", обравши ПІБ із заявки, та оновити таблицю "Слухачі"
& Слухач відображається в таблиці з новим ПІБ
& Failed
\\ \hline \tcn
& Перейти в таблицю "Слухачі", видалити слухача Шшпу Ртдм Стут, зберегти зміни та оновити список
& Слухач відсутній у списку
& 
\\ \hline \tcn
& Відкрити таблицю коефіціентів
& З'явилася таблиця
& 
\\ \hline \tcn
& Запам'ятати значення коефіціенту зарплати персоналу, встановити його у 1.3, зберегти зміни та оновити таблицю
& Коефіціент змінився, дані не пошкоджено
& 
\\ \hline \tcn
& Відновити значення коефіціенту зарплати персоналу, зберегти зміни та оновити таблицю
& Коефіціент змінився, дані не пошкоджено
& 
\\ \hline \tcn
& Вийти з системи
& Відкрився екран входу
& 
\\ \hline \tcn
& Увійти з логіном test111 та паролем test111
& Відобразилося меню системи
& 
\\ \hline \tcn
& Увійти з логіном test111 та паролем test111
& Відобразилося меню системи
& 
\\ \hline \tcn
& Вийти з системи, увійти з логіном asdf та паролем adsf, відкрити таблицю користувачів
& З'явилася таблиця
& 
\\ \hline \tcn
& Скинути пароль для користувача test111, вийти з системи, увійти з логіном test111 та паролем test222
& Відобразилося меню системи
& 
\\ \hline \tcn
& Вийти з системи, увійти з логіном asdf та паролем adsf, відкрити таблицю користувачів, видалити користувача test111, зберегти зміни та оновити таблицю
& Користувач test111 зник зі списку
& 
\\ \hline \tcn
& Скинути пароль для користувача test111, вийти з системи, увійти з логіном test111 та паролем test222
& Система відхилює авторизацію
& 
\\ \hline \tcn
& Увійти з логіном test222 та паролем test222
& Система відхилює авторизацію
& 
\\ \hline
\end{supertabular}
