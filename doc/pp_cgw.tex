\documentclass[a4paper,14pt]{article}
%\usepackage{showframe}
\usepackage{indentfirst}
\usepackage{extsizes}
\usepackage{fontspec}
\usepackage{multirow}
\usepackage{amssymb}
\usepackage[rm,center,uppercase,tiny]{titlesec}
\usepackage{standalone}
\usepackage{hyperref}
\usepackage{supertabular}
\usepackage{tikz}
\usetikzlibrary{mindmap}
\usepackage[below]{placeins}
\usepackage[absolute,overlay]{textpos}
\usepackage{diagbox}
\usepackage{ragged2e}
\usepackage{lscape}
\usetikzlibrary{shapes, arrows, chains}
\usepackage[left=2cm, top=2cm, right=2cm, bottom=4cm, nohead]{geometry}
\titleformat{\section}{\centering}{}{0pt}{\setcounter{tabcount}{0}\thesection\ \uppercase}{}
\titleformat{\subsection}{\centering}{}{0pt}{\thesubsection\ \uppercase}{}
\usepackage{polyglossia}
\setlength{\headsep}{1cm}
\setlength{\parindent}{1cm}
\setmainfont{Times New Roman}
\setmonofont{Courier New}
\setdefaultlanguage{ukrainian}
\setotherlanguage{english}
\newfontfamily\cyrillicfont[Mapping=tex-text]{Times New Roman}
\newfontfamily\cyrillicfonttt{Anonymous Pro}
\begin{document}
\newcounter{imgcount}
\newcounter{addimgcount}
\newcounter{tabcount}
\def\imglabel#1{\addtocounter{imgcount}{1} \begin{center}Рисунок \thesection .\theimgcount\ --- #1 \end{center}}
\def\addimglabel#1#2{\addtocounter{addimgcount}{1} \begin{center}Рисунок #1.\theaddimgcount\ --- #2 \end{center}}
\def\addpolyimglabel#1#2#3{\addtocounter{addimgcount}{1} \begin{center}Рисунок #1.\theaddimgcount, аркуш #3\ --- #2 \end{center}}
\def\tablabel#1{\addtocounter{tabcount}{1} \vspace{\baselineskip}\par Таблиця \thesection .\thetabcount\ --- #1}

\tablefirsthead{}
\tablelasttail{}
\tabletail{\hline}

\pagestyle{empty}
\makeatletter
\renewcommand{\@oddfoot}{\raisebox{-1cm}{\parbox[b]{\textwidth}{
\fontsize{10pt}{12pt} \selectfont
Посилання на цей документ: \url{https://github.com/bodqhrohro/knp2014/raw/master/doc/pp_cgw.pdf}
}}}
\makeatother

\begin{center}
Одеський національний політехнічний університет

Інститут комп'ютерних систем

Кафедра системного програмного забезпечення
\vspace{55mm}

Пояснювальна записка

до Розрахунково-графічної роботи

з дисципліни <<Проектний практикум>>

на тему <<Система обліку платних курсів. Модуль обліку платних курсів>>
\end{center}
\vspace{5cm}
\begin{flushright}
Виконав: ст. гр. АС-122 Горбешко Б. М.

Перевірив: Блажко О. А.
\end{flushright}
\vspace{55mm}
\begin{center}Одеса --- 2016 року\end{center}
\newpage
\makeatletter\renewcommand{\@oddhead}{\hfil\thepage}\makeatother

\def\contentsname{ЗМІСТ}
\tableofcontents
\newpage

\section{Визначення бізнес-вимог}
\begin{tikzpicture}
\path[
 mindmap,
 scale=0.8,
 concept color=white,
 level 1 concept/.append style={sibling angle=120},
 level 2 concept/.append style={sibling angle=60},
 level 3 concept/.append style={sibling angle=30},
 sys/.style={concept, font=\footnotesize}
]
node [sys] {Система обліку платних курсів\\із підтримкою самостійного запису слухачів через мережу Інтернет} [clockwise from=210]
 child {
  node {Вхідний потік} [clockwise from=330]
   child { node {Хто?} [clockwise from=340]
    child { node {Оператори} }
    child { node {Адміністратор} }
    child { node {Слухачі}
     child { node {Студенти} }
     child { node {Сторонні} }
    }
   }
   child { node {Що?} [clockwise from=320]
    child { node {Дані слухачів} }
    child { node {Курси} }
    child { node {Дані викладачів} }
    child { node {Коефіціенти} }
    child { node {Розклад курсів} }
   }
   child { node {Де?} [clockwise from=200]
    child { node {Web-сайт} }
    child { node {Кафедра СПЗ} }
   }
   child { node {Коли?}  [clockwise from=180]
    child { node {Початок чверті} }
    child { node {Довільно} }
   }
   child { node {Як?} [clockwise from=150]
    child { node {Ручне введення} }
    child { node {Імпорт з системи електронного деканату} }
   }
 } child {
  node {Внутрішній потік} [clockwise from=210]
   child { node {Хто?} }
   child { node {Що?} 
    child { node {Збереження даних} }
    child { node {Контроль цілісності} }
    child { node {Розрахунок вартості курсу} }
   }
   child { node {Де?} [clockwise from=105]
    child { node {Web-сервер} }
    child { node {СКБД} }
   }
   child { node {Коли?} [clockwise from=60]
    child { node {Постійно} }
   }
   child { node {Як?} [clockwise from=30]
    child { node {Формула} }
   }
 } child {
  node {Вихідний потік} [clockwise from=90]
   child { node {Хто?} [clockwise from=120]
    child { node {Викладачі} }
    child { node {Оператори} }
    child { node {Адміністратор} }
    child { node {Слухачі} [clockwise from=10]
     child { node {Студенти} }
     child { node {Сторонні} }
    }
   }
   child { node {Що?} [clockwise from=35]
    child { node {Сповіщення} }
    child { node {Звіти} 
     child { node {За період} }
     child { node {За курсом} }
    }
    child { node {Розклад} }
   }
   child { node {Де?} [clockwise from=30]
    child { node {Web-сайт} }
    child { node {Кафедра СПЗ} }
    child { node {Бухгалтерія} }
   }
   child { node {Коли?} [clockwise from=340]
    child { node {Кінець семестру} }
    child { node {Під час курсів} }
   }
   child { node {Як?} [clockwise from=310]
    child { node {Звіти} 
     child { node {У файл} }
     child { node {Друк} }
    }
    child { node {Сповіщення} [clockwise from=280]
     child { node {На e-mail} }
     child { node {На сайті} }
    }
   }
 };
\end{tikzpicture}

\bigbreak
\subsection{Функціональні вимоги}
\bigbreak
Узагальнені варіанти використання включено до функціональних вимог та проаналізовано за принципом MoSCoW [7]:
\begin{itemize}
 \item Життєвий цикл
 \begin{itemize}
  \item Організаційні процеси
  \begin{itemize}
   \item Навчання
   \begin{itemize}
    \item Опанування Kendo UI
   \end{itemize}
   \item Створення інфраструктури
   \item Керування проектом
  \end{itemize}
  \item Основні процеси
  \begin{itemize}
   \item Розробка
   \begin{itemize}
    \item Аналіз вимог
    \item {[}M{]} ВВ: реєстрація
    \item {[}M{]} ВВ: авторизація
    \item {[}M{]} ВВ: робота з викладачами
    \item {[}M{]} ВВ: робота з курсами
    \item {[}W{]} ВВ: робота з розкладом
    \item {[}M{]} ВВ: робота зі слухачами
    \item {[}M{]} ВВ: запис слухачів на курс
    \item {[}M{]} ВВ: робота з обліковими записами
    \item {[}C{]} ВВ: робота з коефіціентами
    \item {[}S{]} ВВ: створення звітів
    \item {[}S{]} ВВ: обробка заявок
    \item {[}M{]} ВВ: обробка облікових записів
   \end{itemize}
   \item Впровадження
  \end{itemize}
  \item Допоміжні процеси
  \begin{itemize}
   \item Документування
   \item Забезпечення якості
   \begin{itemize}
    \item Модульне тестування
    \item Тестування інтерфейсу користувача
   \end{itemize}
   \item Вирішення проблем
  \end{itemize}
 \end{itemize}
\end{itemize}
\newpage
\subsection{Нефункціональні вимоги}
\bigbreak
\begin{itemize}
 \item Відображати та редагувати дані, що вводяться в систему, у таблицях
 \item Відображати на екрані одну або дві різні таблиці одночасно
 \item Вкладені таблиці для сутностей, пов'язаних зв'язком M--M
 \item Гарячі клавіші для додання нового запису, видалення запису, збереження запису
 \item Підтримка стабільних версій браузерів Firefox та Chrome, Internet Explorer 8--11
 \item Час реакції $\leqslant$ 5 секунд
 \item Імовірність збою --- 0.01
 \item Підтримка резервного копіювання даних та відновлення з резервних копій
\end{itemize}
\bigbreak
\subsection{Планування розробки}
\bigbreak
Представлення: клієнт. Бізнес-логіка: сервер. Дані: БД.

Технології розробки: мова PHP5.4, СКБД MySQL 5, Web-технології, формат даних JSON.

Інструменти розробки: редактор Vim, система контролю версій Git, бібліотека тестування запитів Dogpatch.

Діаграми: рис. А.3, А.4.

\newpage
\section{Проектування програмного продукту}
\subsection{Концептуальне проектування}
\bigbreak
Діаграму концептуальних класів наведено на рис. 2.1.

\noindent\includegraphics[width=17cm]{pp_pw3_conc.png}
\imglabel{Діаграма концептуальних класів}

Клас <<Курс>> характеризується високою зв'язністю [8], тобто це головний клас у системі.
\bigbreak
\subsection{Логічне проектування}
\bigbreak
Діаграму програмних класів наведено на рис. А.5.

Як і на діаграмі концептуальних класів, клас Course характеризується високою зв'язністю та має найбільшу кількість атрибутів. Також високу зв'язність має AccessControl, що відповідає висновку з пункту 1.2.1.

\bigbreak
\subsection{База даних}
\def\dbhead{
 \hline
 \Centering Ключ &
 \Centering Назва &
 \Centering Ім'я поля &
 \Centering Тип &
 \Centering NULL &
 \Centering Дод.\\
 \hline
}
\tablefirsthead{\dbhead}
\tablehead{\multicolumn{4}{l}{Продовження таблиці \thesection .\thetabcount} \\\hline\dbhead}
\def\dbtable#1#2#3{
 \tablabel{Опис структури таблиці <<#1>> (#2)}\\
 \begin{supertabular}{|p{1.2cm}|p{3.5cm}|p{3cm}|p{2.8cm}|p{1.2cm}|p{4.3cm}|}
 #3
 \hline
 \end{supertabular}
 \\[-3mm]
}
\newcommand{\tabheader}[2]{
 #1 \emph{(#2)}
}
\setlength{\tabcolsep}{3pt}
\dbtable{Слухачі}{listeners}{
PK & Ідентифікатор & idListener & int(6) & Ні & A\_I\\
 & Ім'я & Name & varchar(30) & Так & \\
 & Прізвіще & Surname & varchar(30) & Ні & \\
 & По батькові & Patronymic & varchar(30) & Так & \\
 & Університетська група & UGroup & varchar(6) & Так & \\
 & Телефон & Phone & varchar(20) & Так & \\
 & E-mail & Email & varchar(40) & Так & \\
FK & Ким змінено & affectedBy & varchar(32) & Ні & \\
}
\dbtable{Курси}{courses}{
PK & Ідентифікатор & idCourse & int(6) & Ні & A\_I\\
 & Назва & Name & varchar(50) & Ні & \\
 & Опис & Description & varchar(400) & Так & \\
 & Ознака індивідуального курсу & isIndividual & bit(1) & Ні & \\
FK & Викладач & idTeacher & int(6) & Так & \\
FK & Другий викладач & idTeacher2 & int(6) & Так & \\
 & Ціна & Price & decimal(10,2) & Так & \\
 & Стан & state & tinyint(1) & Ні & 0..2 (Йде набір, набрано, завершений)\\
FK & Ким змінено & affectedBy & varchar(32) & Ні & \\
}
\dbtable{Слухачі курсу}{Course\_Listeners}{
PK & Ідентифікатор & idCL & int(6) & Ні & A\_I\\
FK & Курс & idCourse & int(6) & Ні & \\
FK & Слухач & idListener & int(6) & Ні & \\
 & Оцінка & mark & tinyint(3) & Так & \\
FK & Ким змінено & affectedBy & varchar(32) & Ні & \\
}
\dbtable{Платежі}{payments}{
PK & Ідентифікатор & idPayment & int(6) & Ні & A\_I\\
FK & Слухач курсу & idCL & int(6) & Ні & \\
 & Кошти & delta & int(6) & Ні & \\
 & Мітка часу & timestamp & timestamp & Ні & CURRENT\_\newline TIMESTAMP\\
}
\dbtable{Викладачі}{teachers}{
PK & Ідентифікатор & idTeacher & int(6) & Ні & A\_I\\
 & Ім'я & Name & varchar(30) & Так & \\
 & Прізвіще & Surname & varchar(30) & Ні & \\
 & По батькові & Patronymic & varchar(30) & Так & \\
 & Телефон & Phone & varchar(20) & Так & \\
 & E-mail & Email & varchar(40) & Так & \\
FK & Ким змінено & affectedBy & varchar(32) & Ні & \\
FK & Вчений ступінь & degree & tinyint(2) & Так & \\
}
\dbtable{Розцінки}{prices}{
PK & Ідентифікатор & degree & tinyint(2) & Ні & A\_I\\
 & Вчений ступінь & deglab & varchar(30) & Ні & \\
 & Зарплата & salary & decimal(10,2) & Ні & \\
}
\dbtable{Коефіціенти}{coefficients}{
PK & Назва & name & varchar(64) & Ні & \\
 & Мітка & label & varchar(128) & Так & \\
 & Значення & value & float & Так & \\
}
\dbtable{Заняття}{lessons}{
PK & Ідентифікатор & idLesson & int(11) & Ні & A\_I\\
FK & Курс & idCourse & int(11) & Ні & \\
 & Дата & date & date & Так & \\
 & Час початку & time & time & Ні & \\
 & Тип & type & tinyint(1) & Ні & 0..1 (Лекція, Практика)\\
}
\dbtable{Користувачі}{users}{
PK & Логін & login & varchar(32) & Ні & \\
 & Хеш паролю & password & text & Ні & \\
 & Сіль & salt & text & Ні & \\
 & Тип & type & tinyint(1) & Ні & 0..2 (Оператор, Адміністратор, Переглядач)\\
 & Ключ сесії & sessionid & text & Так & \\
}
\bigbreak
\subsection{Оцінка алгоритмічної складності}
\bigbreak
На рис. А.6 зазначено складну операцію розрахунку вартості за даними з БД. Вона реалізується наступним SQL-запитом:

{ \fontsize{11pt}{12pt} \selectfont
\begin{verbatim}
SELECT ifnull(s1,0)*ifnull(c1,0)+ifnull(s2,0)*ifnull(c2,0) FROM ((
 SELECT salary as s1 FROM prices WHERE degree IN (
  SELECT degree FROM teachers WHERE idTeacher IN (
   SELECT idTeacher FROM courses WHERE idCourse=?
))) s1 LEFT OUTER JOIN (
 SELECT salary as s2 FROM prices WHERE degree IN(
  SELECT degree FROM teachers WHERE idTeacher IN (
   SELECT idTeacher2 FROM courses WHERE idCourse=?
))) s2 ON 1=1) LEFT OUTER JOIN ((
 SELECT count(*) as c1 FROM lessons WHERE idCourse=? AND type=0
) c1 ON 1=1 LEFT OUTER JOIN (
 SELECT count(*) as c2 FROM lessons WHERE idCourse=? AND type=1
) c2) ON 1=1;
\end{verbatim}
}

\FloatBarrier
План виконання запиту:
\begin{enumerate}
\item Вибірка вмісту таблиці courses
\item Проекція кортежів за значенням атрибуту idCourse
\item Проекція атрибутів за атрибутом idCourse
\item Вибірка вмісту таблиці teachers
\item Проекція кортежів за значенням атрибуту idTeacher
\item Проекція атрибутів за атрибутом degree
\item Вибірка вмісту таблиці prices
\item Проекція кортежів за значенням атрибуту degree
\item Проекція атрибутів за атрибутом salary
\item Використання вибірки з п. 1.
\item Проекція кортежів за значенням атрибуту idCourse
\item Проекція атрибутів за атрибутом idCourse
\item Використання вибірки з п. 4.
\item Проекція кортежів за значенням атрибуту idTeacher
\item Проекція атрибутів за атрибутом degree
\item Використання вибірки з п. 7.
\item Проекція кортежів за значенням атрибуту degree
\item Проекція атрибутів за атрибутом salary
\item Вибірка вмісту таблиці lessons
\item Проекція кортежів за значенням атрибуту idCourse
\item Проекція кортежів за значенням атрибуту type
\item Підрахунок кортежів
\item Використання проекції з п. 20.
\item Проекція кортежів за значенням атрибуту type
\item Підрахунок кортежів
\item З'єднання результатів пп. 9 та 18.
\item З'єднання результатів пп. 22 та 25.
\item З'єднання результатів пп. 26 та 27.
\item Розрахунок математичного виразу за результатом п. 28.
\end{enumerate}

Операції проекції кортежів у пп. 2, 3, 5, 6, 8, 9, 11, 12, 14, 15, 17 та 18 повинні повертати один кортеж. Пошук у таблицях courses та teachers відбувається за первинним ключем, а таблиця prices складається з кількох кортежів. Для проекцій з пп. 20, 21 та 24 доцільно створити у таблиці lessons індекс за атрибутом idCourse. За атрибутом type індекс недоцільний, оскільки проекції за ним відбуваються не з усієї таблиці, а можливих значення тільки два. SQL-запит для створення індексу:

{ \fontsize{11pt}{12pt} \selectfont
\begin{verbatim}
CREATE INDEX course_of_lesson ON lessons (idCourse);
\end{verbatim}
}

Циклів в алгоритмах на рис. А.6 та А.7 немає, тож їх складність --- \emph{O(1)} [9].

\bigbreak
\subsection{Опис зовнішних інтерфейсів}
\bigbreak
Інтерфейс користувача побудовано із використанням бібліотеки Kendo UI, дизайн визначається наявними для неї темами оформлення. Доступні самостійні концептуальні класи присутні у головному меню та розташовані за частотою використання: найактивніша робота проводиться зі слухачами та курсами, викладачі та користувачі заповнюються на початку роботи і потім змінюються рідко, звіти здаються рідко, розцінки та коефіціенти змінюються у виключних випадках. В кінці головного меню розміщено кнопку виходу з системи. Екран авторизації є окремим та лаконічним, містить лише форму та кнопку перемикання форм реєстрації та входу.

Для роботи серверної частини системи потрібен веб-сервер Apache 2.x, інтерпретатор PHP5 не нижче 5.4, СКБД MySQL або MariaDB 5. Для користування клієнтською частиною потрібен web-браузер із підтримкою EcmaScript 5; тестування проводиться у поточних версіях браузерів Mozilla Firefox та Chromium для десктопу.

Система може працювати у межах однієї машини, через локальну мережу та через мережу Інтернет, в залежності від мережевих підключень та налаштувань машини, на якій встановлено серверну частину. Підключення, відповідно, може здійснюватись будь-яким доступним дротовим або бездротовим каналом підключення до локальної мережі або мережі Інтернет: Ethernet, Wi-Fi, ADSL, HSPA, Dial-up та ін. Оскільки звіти генеруються у форматі PDF, для їх перегляду потрібна програма --- переглядач PDF: Mozilla Firefox, Google Chrome, Adobe Reader, Foxit Reader тощо. [10] Для друку потрібен принтер, підключений безпосередньо до машини, на якій відкрито файл звіту, або доступний для неї через мережу.

\section{Конструювання програмного продукту}
\newpage
\section{Конструювання програмного продукту}
\subsection{Опис програмних технологій}
\bigbreak
Система поділяється на дві частини: сервер та клієнт.

Сервер реалізовано на мові PHP5. [11] Це скриптова мова програмування, орієнтована на задачі web-розробки. Завдяки відсутності етапу компіляції зменшується час розробки та відлагодження. Динамічна типізація мови дозволяє прозоро працювати як з числами, так і з їх текстовим представленням, а також спрощує перевірку наявності об'єктів у булевих виразах. Мова надає вбудовані засоби для обробки рядків, параметрів HTTP-запитів та взаємодії з різними СКБД. Окрім цього, PHP є єдиною підтримуваною мовою на багатьох хостингах для web-сайтів та web-застосунків. З цієї ж причини у якості СКБД обрано MySQL 5. Задля використання нових можливостей, зокрема, скороченого оголошення масивів, мінімальною підтримуваною версією є PHP 5.4.

Клієнт реалізовано на скриптовій мові програмування ECMAScript 5 (більше відомій як JavaScript) та описових мовах HTML 5 та CSS 3. Вони є фактичними web-стандартами та підтримуються сучасними web-браузерами, такими як Firefox, Chrome, Edge, Safari, Internet Explorer, Android Browser тощо. Альтернативи не є кросбраузерними, приміром, мова VBScript підтримується лише у Internet Explorer, а мова Dart --- лише у Chrome. Крім того, використання web-технологій дозволяє створювати на основі клієнту застосунки для настільних та мобільних комп'ютерів за допомогою спеціальних середовищ на основі браузерних рушіїв (напр., node-webkit, PhoneGap, AppJS).
\bigbreak
\subsection{Опис програмних бібліотек}
\bigbreak
Оскільки клієнт є <<товстим>>, на сервер покладені лише задачі передачі та обробки даних між клієнтом та базою даних, тож немає потреби у використанні систем керування контентом, фреймворків та тому подібних бібліотек. Проте мова PHP не має вбудованих засобів для побудови PDF-документів, тож для цієї задачі використано бібліотеку MPDF, яка формує документ з HTML-шаблону. Серед альтернатив розглядалася також бібліотеку TCPDF, проте при спробі використання вона виявила недостатню підтримку можливостей HTML та CSS.

Модульні тести для сервеу також написані мовою PHP та використовують бібліотеку Dogpatch, розширену та доповнену для потреб тестування системи. Бібліотека DogPatch дозволяє виконувати HTTP-запити та таким чином перевіряти, чи вірні відповіді та HTTP-коди надають запити за певних умов. Додано функції для обробки відповідей у JSON, що дозволяють перевірити наявність певного об'єкту у масиві об'єктів.

Клієнт використовує бібліотеки jQuery та Kendo UI. jQuery є прошарком над API, що доступні для браузерного JavaScript, який надає компактний синтаксис та кросбраузерні реалізації для таких частовживаних операцій, як робота з DOM та AJAX-запитами. Kendo UI надає елементи керування (віджети) із широкою фукціональністю, у тому числі такі, що не надаються засобами HTML. Крім того, вона надає багатофункціональний елемент керування для представлення та редагування табличних даних, а також самостійно здійснює запити з цими даними до серверу, слід лише описати URL та формати даних. [12] Для призначення гарячих клавіш використовується плаґін для jQuery jquery.hotkeys.
\bigbreak
\subsection{Особливості створення програмних модулів з урахуванням мови програмування}
\bigbreak
\begin{sloppy}
Інтерпретатор та мова PHP5 розраховані на використання у якості CGI, тобто web-сервер використовує URI як відносний шлях до файлу, запускає файл на виконання та віддає результат його роботи. Альтернативні способи використання URI, зокрема, маскування структури програмних модулів за допомогою так званих <<людинозрозумілих URL>> та маршрутизації запитів до викликів методів класів вимагають низькорівневих перехоплень за допомогою правил ModRewrite для Apache та аналогічних засобів для інших web-серверів. Тому натомість архітектуру серверної частини системи реалізовано без застосування об'єктно-орієнтованого програмування, проте із наслідуванням деяких його принципів. Модулі організовані по директоріях за концептуальними класами і реалізують інтерфейси шляхом розміщення у директоріях файлів з однаковими іменами, способами передачі параметрів та форматами вхідних та вихідних даних --- це дозволяє за реалізації роботи з різними концептуальними класами у клієнті змінювати лише назву директорії. Функції для зв'язку між концептуальними класами винесено в окремі модулі, спільні функції винисено в окремі бібліотеки функцій.

Клієнт реалізовано як односторінковий застосунок. Головний інтерфейс системи реалізований HTML-сторінкою з меню; за натисненням пунктів меню створюються та відкриваються певні вікна засобами JavaScript. Проте сторінка входу та реєстрації, як незалежна від головного інтерфейсу сутність, винесена в окрему HTML-сторінку, до якої не підключено JavaScript-бібліотек. Обидві сторінки можуть змінюватисься в залежності від режиму, повідомлень системи, прав доступу тощо, тому виводяться скриптами на мові PHP, що є HTML-шаблонами зі вставками коду на PHP. Власних функцій на JavaScript небагато і слугують вони здебільшого прошарком між описами даних та реалізованими засобами бібліотеки Kendo UI елементами керування, тож зібрані в один модуль. За розвитку системи та додання нових функцій може знадобитися декомпозиція цього модулю.
\end{sloppy}
\bigbreak
\subsection{Особливості створення структур даних}
\bigbreak
У якості формату передачі даних від сервера до клієнту обрано JSON. Це текстовий формат на основі мови JavaScript, що надає компактний безнадлишковий (порівняно з похідними від SGML мовами) синтаксис та дозволяє структурувати рядки та числа у вкладених масивах та об'єктах (асоціативних масивах). [13] JSON формується засобами мови PHP, що доступні з версії 5.1. Дані від клієнта до сервера передаються у тілі POST-запиту у форматі HTML-форм. Структури даних, що використовуються при роботі з бібліотекою Kendo UI, визначаються документацією до цієї бібліотеки.

\subsection{Модульне тестування}

\tablabel{Тестові дані для таблиці courses}
\begin{center}
\begin{tabular}{|l|c|c|}
\hline
Атрибут & \multicolumn{2}{|c|}{Значення} \\
\hline
idCourse & 1 & 2 \\
Name & a & b \\
Description & a & b \\
isIndividual & 0 & 0 \\
idTeacher & 1 & 1 \\
idTeacher2 & 2 & 2 \\
Price & 15 & 0 \\
state & 0 & 0 \\
affectedBy & user1 & user2 \\
\hline
\end{tabular}
\end{center}

\newpage
\tablabel{Тестові дані для таблиці Course\_Listeners}
\begin{center}
\begin{tabular}{|l|c|c|c|c|}
\hline
Атрибут & \multicolumn{4}{|c|}{Значення} \\
\hline
idCL & 1 & 2 & 3 & 4 \\
idCourse & 1 & 1 & 1 & 2 \\
idListener & 1 & 2 & 3 & 1 \\
mark & 0 & 0 & 0 & 0 \\
affectedBy & user1 & user1 & user1 & user1 \\
\hline
\end{tabular}
\end{center}

\tablabel{Тестові дані для таблиці teachers}
\begin{center}
\begin{tabular}{|l|c|c|}
\hline
Атрибут & \multicolumn{2}{|c|}{Значення} \\
\hline
idTeacher & 1 & 2 \\
Name & a & b \\
Surname & a & b \\
Patronymic & a & b \\
Phone & 289 & 2893 \\
Email & b@a.b & c@d.c \\
degree & 1 & 2 \\
affectedBy & user1 & user2 \\
\hline
\end{tabular}
\end{center}

\tablabel{Тестові дані для таблиці prices}
\begin{center}
\begin{tabular}{|l|c|c|c|}
\hline
Атрибут & \multicolumn{3}{|c|}{Значення} \\
\hline
degree & 1 & 2 & 3 \\
deglab & Б. с. & Доц. к. т. н. & Проф. \\
salary & 15.62 & 23.24 & 35.38 \\
\hline
\end{tabular}
\end{center}

\tablabel{Тестові дані для таблиці lessons}
\begin{center}
\begin{tabular}{|l|c|c|c|c|c|c|}
\hline
Атрибут & \multicolumn{6}{|c|}{Значення} \\
\hline
idLesson & 1 & 2 & 3 & 4 & 5 & 6 \\
idCourse & 1 & 1 & 1 & 2 & 2 & 2 \\
date & 1.4.15 & 2.4.15 & 3.4.15 & 4.4.15 & 5.4.15 & 6.4.15 \\
time & 13:00 & 13:00 & 13:00 & 13:00 & 13:00 & 13:00 \\
type & 1 & 1 & 2 & 1 & 2 & 2 \\
\hline
\end{tabular}
\end{center}

\tablabel{Тестові дані для таблиці users}
\begin{center}
\begin{tabular}{|l|c|c|c|}
\hline
Атрибут & \multicolumn{3}{|c|}{Значення} \\
\hline
login & adm1 & user1 & user2 \\
password & $hash('asdf')$ & $hash('qwer')$ & $hash('zxcv')$ \\
salt & $<hash1>$ & $<hash2>$ & $<hash3>$ \\
type & 0 & 1 & 2 \\
sessionid & $<key1>$ & NULL & q \\
\hline
\end{tabular}
\end{center}

{\setlength{\tabcolsep}{0pt}
\tablabel{Тестові дані для таблиці coefficients}
\begin{center}
\begin{tabular}{|l|c|c|c|}
\hline
Атрибут & \multicolumn{3}{|c|}{Значення} \\
\hline
name & bonus & others & personal \\
label & \parbox{4.5cm}{Нарахування на заробітну плату} & Інші послуги та утримки & Зарплата персоналу \\
value & 1.12 & 1.23 & 1.05 \\
\hline
\end{tabular}
\end{center}}

Тестування алгоритму розрахунку вартості курсу проведено методом "білої скрині" (таблиця 3.7), алгоритму аутентифікації та авторизації --- методом "чорної скрині" (таблиця 3.8).
\tablabel{Тестові випадки для алгоритму розрахунку вартості курсу}
\newcounter{tcnt}\def\tcn{\addtocounter{tcnt}{1} \thetcnt}
\begin{tabular}{|c|c|c|c|c|}
\hline
\multirow{2}{23mm}{\centering \textbf{Test Case №}} &
\multicolumn{2}{c|}{\textbf{Вхідні дані}} &
\multirow{2}{27mm}{\centering \textbf{Очікуваний результат}} &
\multirow{2}{77mm}{\centering \textbf{Результат тестування (успішний (passed) / неуспішний (failed))}} \tabularnewline
\cline{2-3}
&id&full&&\tabularnewline
\hline
\tcn& 1	& true	& 60	&	\\
\tcn& 2	& true	& 15	&	\\
\tcn& 1	& false	& 20	&	\\
\tcn& 2	& false	& 15	&	\\
\hline
\end{tabular}
\setcounter{tcnt}{0}
\tablabel{Тестові випадки для алгоритму аутентифікації та авторизації}\\
\begin{tabular}{|c|c|c|c|c|}
\hline
\multirow{2}{18mm}{\centering \textbf{Test Case №}} &
\multicolumn{2}{c|}{\textbf{Вхідні дані}} &
\multirow{2}{27mm}{\centering \textbf{Очікуваний результат}} &
\multirow{2}{75mm}{\centering \textbf{Результат тестування (успішний (passed) / неуспішний (failed))}} \tabularnewline
\cline{2-3}
&login&pass&&\tabularnewline
\hline
\tcn& 'adm1'	& 'asdf'	& Вхід		&	\\
\tcn& 'adm1'	& 'qwer'	& Відмова	&	\\
\tcn& 'adm1'	& NULL		& Відмова	&	\\
\tcn& 'user3'	& 'qwer'	& Відмова	&	\\
\tcn& 'user3'	& 'sdfj'	& Відмова	&	\\
\tcn& 'user2'	& 'zxcv'	& Відмова	&	\\
\hline
\end{tabular}

\subsection{Функціональне тестування}
\tablabel{Тестові випадки для варіантів використання}\\
\newcounter{tcnt}\def\tcn{\addtocounter{tcnt}{1} \thetcnt}
\tablefirsthead{}
\tablelasttail{}
\tablehead{\hline \multicolumn{4}{|l|}{Продовження таблиці \thesection .\thetabcount} \\}
\tabletail{\hline}
\begin{supertabular}{|p{13mm}|p{65mm}|p{45mm}|p{3cm}|}
\hline
\textbf{Test Case №} &
\textbf{Дія} &
\textbf{Очікуваний результат} &
\textbf{Результат тестування (успішний (passed) / неуспішний (failed))} \\
\hline \tcn
& Перейти на сторінку реєстрації
& Відкрилася сторінка реєстрації
& 
\\ \hline \tcn
& Ввести логін test111 та двічі --- пароль test111
& Повідомлення про успішну реєстрацію
& 
\\ \hline \tcn
& Ввести логін test222 та двічі --- пароль test222
& Повідомлення про успішну реєстрацію
& 
\\ \hline \tcn
& Перейти на сторінку входу
& Відкрилася сторінка входу
& 
\\ \hline \tcn
& Ввести логін asdf та пароль asdf
& Система відхилює авторизацію
& 
\\ \hline \tcn
& Ввести логін test111 та пароль test111
& Система відхилює авторизацію
& 
\\ \hline \tcn
& Ввести логін asdf та пароль adsf
& Відкрився головний інтерфейс системи (рядок меню)
& 
\\ \hline \tcn
& Відкрити таблицю викладачів
& З'явилася таблиця
& 
\\ \hline \tcn
& Створити викладача ("Ппшш" "Шррр" "Тссс" "+3820" "ba@b.c"), зберегти зміни та оновити список
& Викладач присутній в списку та дані не пошкоджено
& 
\\ \hline \tcn
& Змінити ім'я щойно створеного викладача на "Шссс", зберегти зміни та оновити список
& Ім'я викладача змінилося, дані не пошкоджено
& 
\\ \hline \tcn
& Відкрити таблицю курсів
& З'явилася таблиця
& 
\\ \hline \tcn
& Створити курс ("" "Шррр" "Ппшш Ш. Т." "29" "Йде набір") та зберегти зміни
& Система вимагає ввести назву курсу
& 
\\ \hline \tcn
& Створити курс ("Ппшш" "Шррр" "Ппшш Ш. Т." "Ппшш Ш. Т." "29" "Йде набір"), зберегти зміни та оновити список
& Курс присутній в списку та дані не пошкоджено
& 
\\ \hline \tcn
& Змінити назву щойно створеного курсу на "Ппрр", зберегти зміни та оновити список
& Назва курсу змінилася, дані не пошкоджено
& 
\\ \hline \tcn
& Відкрити для щойно створеного курсу форму додання заняття
& Відкрилася форма
& 
\\ \hline \tcn
& Створити для щойно створеного курсу заняття ("01.04.2016" "13:30" "Лекція"), зберегти зміни та оновити список
& Заняття присутнє в списку та дані не пошкоджено
& Failed
\\ \hline \tcn
& Змінити час щойно створеного заняття на "13:45", зберегти зміни та оновити список
& Час змінився, дані не пошкоджено
& Failed
\\ \hline \tcn
& Створити для щойно створеного курсу заняття ("02.04.2016" "13:30" "Практика"), зберегти зміни та оновити список
& Заняття присутнє в списку та дані не пошкоджено
& Failed
\\ \hline \tcn
& Видалити заняття за дату 01.04.2016 та оновити список
& Заняття відсутнє у списку
& Failed
\\ \hline \tcn
& Встановити для заняття, що залишилося, тип "Лекція", зберегти зміни та оновити список
& Тип змінився, дані не пошкоджено
& Failed
\\ \hline \tcn
& Відкрити таблицю слухачів
& З'явилася таблиця
& 
\\ \hline \tcn
& Створити слухача ("Шшпп" "Ршшш" "Сттт" "АЯ-313" "+3289" "db@d.b" ""), зберегти зміни та оновити список
& Слухач присутній в списку та дані не пошкоджено
& 
\\ \hline \tcn
& Змінити ім'я щойно створеного слухача на "Ртдм", зберегти зміни та оновити список
& Ім'я слухача змінилося, дані не пошкоджено
& 
\\ \hline \tcn
& Розгорнути для щойно створеного слухача підтаблицю запису на курси
& Розгорнулася підтаблиця
& 
\\ \hline \tcn
& Записати слухача на курс "Ппрр"
& Курс з'явився в підтаблиці
& 
\\ \hline \tcn
& Перейти в таблицю "Курси"
& Таблиця перекрила інші таблиці
& 
\\ \hline \tcn
& Розгорнути для курсу "Ппрр" підтаблицю слухачів
& Підтаблиця розгорнулася і містить слухача "Шшпп Р. С."
& 
\\ \hline \tcn
& Встановити слухачеві "Шшпп Р. С." оплату "29", зберегти зміни та оновити список
& Оплата 29 грн., боргу немає
& 
\\ \hline \tcn
& Відкрити форму створення звіту за період
& Відкрилася форма
& 
\\ \hline \tcn
& Обрати дати 29.02.2016 та 28.02.2016 і створити звіт
& Система не приймає дати
& 
\\ \hline \tcn
& Обрати дати 29.02.2037 та 01.03.2037 і створити звіт
& Система не приймає дати
& 
\\ \hline \tcn
& Обрати дати 01.04.2016 та 02.04.2016 і створити звіт
& Відкривається або завантажується PDF-файл, що містить дані, відповідні даним у таблицях інтерфейсу
& 
\\ \hline \tcn
& Видалити заняття для курсу "Ппрр"
& Курс "Ппрр" не містить занять
& Failed
\\ \hline \tcn
& Видалити курс "Ппрр", зберегти зміни та оновити список
& Курс відсутній у списку
& 
\\ \hline \tcn
& Перейти в таблицю "Викладачі", видалити викладача Ппшш Шссс Тссс, зберегти зміни та оновити список
& Викладач відсутній у списку
& 
\\ \hline \tcn
& Створити у зовнішній системі три заявки, при цьому третю --- на ПІБ "Шшпу Ртдм Стут", та зачекати 20 секунд
& Лічильник сповіщень збільшився на 3
& 
\\ \hline \tcn
& Відкрити сповіщення
& Відображається панель зі сповіщеннями
& 
\\ \hline \tcn
& Підтвердити реєстрацію облікового запису test111
& Сповіщення зникло зі списку
& 
\\ \hline \tcn
& Відхилити реєстрацію облікового запису test222
& Сповіщення зникло зі списку
& 
\\ \hline \tcn
& Підтвердити першу щойно додану заявку та перейти до таблиці "Слухачі"
& Сповіщення зникло зі списку сповіщень, слухач із заявки з'явився в таблиці
& 
\\ \hline \tcn
& Відхилити другу щойно додану заявку та оновити таблицю "Слухачі"
& Сповіщення зникло зі списку сповіщень та не з'явилося в таблиці
& 
\\ \hline \tcn
& Злити третю щойно додану заявку з запропонованим "Шшпп Ртдм Сттт", обравши ПІБ із заявки, та оновити таблицю "Слухачі"
& Слухач відображається в таблиці з новим ПІБ
& Failed
\\ \hline \tcn
& Перейти в таблицю "Слухачі", видалити слухача Шшпу Ртдм Стут, зберегти зміни та оновити список
& Слухач відсутній у списку
& 
\\ \hline \tcn
& Відкрити таблицю коефіціентів
& З'явилася таблиця
& 
\\ \hline \tcn
& Запам'ятати значення коефіціенту зарплати персоналу, встановити його у 1.3, зберегти зміни та оновити таблицю
& Коефіціент змінився, дані не пошкоджено
& 
\\ \hline \tcn
& Відновити значення коефіціенту зарплати персоналу, зберегти зміни та оновити таблицю
& Коефіціент змінився, дані не пошкоджено
& 
\\ \hline \tcn
& Вийти з системи
& Відкрився екран входу
& 
\\ \hline \tcn
& Увійти з логіном test111 та паролем test111
& Відобразилося меню системи
& 
\\ \hline \tcn
& Увійти з логіном test111 та паролем test111
& Відобразилося меню системи
& 
\\ \hline \tcn
& Вийти з системи, увійти з логіном asdf та паролем adsf, відкрити таблицю користувачів
& З'явилася таблиця
& 
\\ \hline \tcn
& Скинути пароль для користувача test111, вийти з системи, увійти з логіном test111 та паролем test222
& Відобразилося меню системи
& 
\\ \hline \tcn
& Вийти з системи, увійти з логіном asdf та паролем adsf, відкрити таблицю користувачів, видалити користувача test111, зберегти зміни та оновити таблицю
& Користувач test111 зник зі списку
& 
\\ \hline \tcn
& Скинути пароль для користувача test111, вийти з системи, увійти з логіном test111 та паролем test222
& Система відхилює авторизацію
& 
\\ \hline \tcn
& Увійти з логіном test222 та паролем test222
& Система відхилює авторизацію
& 
\\ \hline
\end{supertabular}

\section{Розгортання програмного продукту}
\subsection{Інструкція з встановлення}
\begin{sloppy}
\begin{enumerate}
\item Встановити та налаштувати web-сервер Apache, інтерпретатор PHP5 не нижче версії 5.4, СКБД MySQL 5 або MariaDB 5. Для запуску сервера та клієнта на одній машині можна скористатися XAMPP з \url{https://www.apachefriends.org/ru/index.html}.
\item Скачати та розпакувати архів: \url{https://github.com/bodqhrohro/knp2014/archive/master.zip}.
\item Розпакувати вміст каталогу src з архіву до каталогу сторінок сайту. за необхідності налаштувати домен.
\item Встановити PHPMyAdmin (\url{https://www.phpmyadmin.net/}), перейти на вкладку "Імпорт" та вибрати файл deploy.sql з каталогу src. Впевнитися, що операція імпорту пройшла успішно.
\item Відкрити у текстовому редакторі файл config.php з кореневого каталогу сайту та вказати там IP чи домен СКБД, логін та пароль для доступу до БД, а також домен, на якому розміщено сайт. Зберегти файл.
\item Відкрити домен, на якому розміщено сайт, у браузері. Впевнитися, що відображається форма входу.\\\includegraphics[width=17cm]{scrns/login.png}\imglabel{Форма входу}
\end{enumerate}
\end{sloppy}
\subsection{Інструкція з використання}
Демонстраційну копію системи розміщено на \url{http://php-bodqhrohro.rhcloud.com/knp2014/}.

В щойно встановленій системі є чотири тестових користувачі: адміністратор asdf, оператор fdsa, переглядачі pnd та qwer. Паролі для них, відповідно: adsf, fdsa, pnd, qwer. Можна запрошувати створення нових користувачів. Натисніть кнопку "Реєстрація" у верхньому правому кутку. введіть логін нового користувача та пароль. Підтвердіть форму, перейдіть знов на сторінку входу та спробуйте увійти під користувачем asdf. Відкриється головний інтерфейс системи --- стрічка меню.
\\\includegraphics[width=17cm]{scrns/menu.png}\imglabel{Головне меню}

При реєстрації нового користувача адміністратори отримують сповіщення. Відкрийте з головного меню панель сповіщень та ввімкніть режим "Детально".
\begin{center}\includegraphics[width=10cm]{scrns/notifications.png}\end{center}\imglabel{Сповіщення}

Можна підтвердити користувача, після чого під ним можна буде зайти, або видалити його. Таким же чином оброблюються заявки слухачів на запис на курси. Щоб сховати панель, натисніть пункт "Сповіщення" ще раз.

Пункт "Вихід" головного меню викликає завершення сесії користувача та перехід на екран входу. Підменю "Звіти" містить пункти, що відкривають форми налаштування звітів. Решта пунктів відкривають таблиці. Натисніть, приміром, пункт "Слухачі". Відкриється таблиця для роботи зі слухачами.

Можна додавати нові рядки за допомогою кнопки на панелі інструментів, видаляти їх за допомогою кнопки зправа та редагувати, клацнувши на потрібній комірці. Змінені комірки підсвічуються. Всі зміни в таблиці не синхронізуються із сервером автоматично --- для цього слугує кнопка "Зберегти зміни". Якщо вміст якоїсь комірки перешкоджає збереженню, вона залишиться підсвіченою.

Таблиці "Слухачі" та "Курси" мають у кожному рядку підтаблиці, що дозволяють записувати слухачів на курси та працювати з оплатами. Розгортаються підтаблиці клацанням по трикутнику з лівого краю рядка. У підтаблицях не можна безпосередньо редагувати комірки, оплата вводиться у формі, що викликається кнопкою "Змінити". При поверненні грошей слухачеві вводиться від'ємна оплата. Додаються слухачі або групи з випадаючого списку; для слухачів він представлений у вигляді дерева з прапорцями, що дозволяє зручно додавати на курс цілі університетські групи та шукати слухачів за групами замість довгого алфавітного списку чи форми пошуку. Слухачі не з університету або з невідомої групи відображаються у піддереві "Інші".

Коли треба згенерувати звіт, виберіть потрібний звіт з випадаючого меню. За необхідності введіть діапазон дат та натисніть кнопку "Створити звіт". Якщо звіт не відкривається у браузері, його можна зберегти та відкрити будь-яким переглядачем PDF та з нього ж відправити на друк.

\end{document}
