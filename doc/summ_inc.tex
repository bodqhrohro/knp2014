\newpage
\section*{Реферат}
\bigbreak
Пояснювальна записка до дипломної роботи: 71 с., 11 рис., 46 табл., 2 додатки, 13 джерел.

Метою створення системи є зменшення часу, що витрачається працівниками кафедри на пов'язану з курсами роботу шляхом автоматизації набору слухачів на курси, реєстрацій сплат та формування звітності, а також для надання майбутнім слухачам можливості ознайомитися з наявними курсами та залишати заявки на реєстрацію за допомогою зовнішніх програмних сервісів.

Методи розробки базуються на мовах програмування PHP та JavaScript, сервері баз даних MySQL, Web-сервері Apache та бібліотеці Kendo UI.

Як результат роботи виконана програмна реалізація системи обліку платних курсів для кафедри СПЗ із можливістю створення звітів та підключення зовнішніх сервісів для прийому заявок.

Ключові слова: RIA, Apache, PHP, MySQL, бухгалтерський облік, JavaScript, Kendo UI.
\newpage
\section*{Abstract}
The aim of work is to decrease the time spent for the courses-related job by the department staff by means of automation of picking listeners for courses, payment enrolling and reports making. Also it should enable potential listeners with the use of external software services to get informed with avaiable courses and to submit applications for registration.

Methods of developing technology are based on PHP and JavaScript programming languages, MySQL database server, Apache Web server and Kendo UI library.

Results: the software realization of a paid course accounting system is completed and the system supports creating reports and connecting external services for the accepting of applications.

Keywords: RIA, Apache, PHP, MySQL, accounts management, JavaScript, Kendo UI.
