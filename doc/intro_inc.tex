\iffalse
\bigbreak
\section*{Скорочення}
\bigbreak
\begin{description}
\setlength{\itemsep}{0pt}
\item[Курс] --- послідовність занять з певної дисципліни у певний період з певним набором слухачів.
\item[Оператор] --- особа, що займається записом слухачів, внесенням курсів у систему, прийомом оплат та формуванням звітів.
\item[Слухач] --- особа, що відвідує курс.
\item[Звіт] --- паперовий чи електронний документ, що містить зведену інформацію про курси.
\item[Розклад] --- перелік дат та часу проведення занять зі вказанням типу заняття та місця його проведення.
\item[Заняття] --- одна урочна година.
\item[Заявка] --- вияв наміру потенційного слухача сплачувати та відвідувати курс.
\item[Коефіціент] --- множник, що використовується при автоматичному розрахунку зарплати викладача і вартості курсу, що залежить від неї.
\item[Сповіщення] --- текстова нотатка, що інформує користувача системи про якусь подію і дозволяє вибрати щодо неї певну дію.
\item[Форма договору] --- бланк документу, що затверджує зобов'язання кафедри СПЗ провести для слухача курс, а слухача --- сплатити курс.
\item[Оплата] --- разове внесення коштів за курс, може не покривати вартість курсу повністю.
\item[Заборгованість] --- різниця між вартість курсу та сумою сплачених слухачем за курс коштів.
\end{description}
\fi
\newpage
\section*{Вступ}
\addcontentsline{toc}{section}{Вступ}
\bigbreak
\begin{sloppy}
На сьогодні автоматизація документообороту так же необхідна, як автоматизація бухгалтерського обліку в середині 90-х років. Причин цьому багато. По-перше, інформацію необхідно обробляти якомога швидше та якісніше, інколи інформаційні потоки не менш важливі, ніж матеріальні. По-друге, втрата інформації чи її потрапляння в чужі руки може дуже дорого обійтися. Можна виділити ряд спільних для організацій, де робота з документами ведеться традиційним способом, проблем:
\begin{itemize}
\item документи губляться;
\item накопичуються документи, призначення та джерело яких незрозумілі;
\item документи та інформація, що в них знаходиться, потрапляє в чужі руки;
\item витрачається багато часу на пошук потрібного документу;
\item на підготовку і узгоду документів витрачається багато часу.
\end{itemize}

Автоматизація документообігу необхідна в будь-якій організації, незалежно від масштабу та типу власності. Але програми не можуть бути універсальними, покривати весь спектр функціональності різних організацій, при цьому залишаючись простими у використанні, надійними в роботі та доступними у фінансовому плані. Універсальність системи призводить до ускладнень у роботі системи, оскільки така система повинна забезпечити, окрім основної функціональності, ще й засоби конфігурації під певну предметну область.

Керування навчальними курсами, що проводяться на кафедрі ВНЗ, обов'язково супроводжується документами, кількість яких із часом зростає. Крім того, проводиться різноманітний аналіз наявних даних, що потребує деяких витрат часу, наприклад, підготовка звіту про боржників, підсумків успішності студентів. Тому створення ефективної системи документообігу з максимальної віддачею та мінімальними затратами є важливим завданням.

Наразі існує багато програмних продуктів, створених з метою автоматизації бізнес-процесів на підприємствах, але далеко не завжди університет має змогу придбати інформаційну систему необхідного рівня, не кажучи про те, що впровадження сторонніх розробок й адаптація програмного продукту до особливостей певної організації завжди породжує багато проблем.

У перспективі застосування систем для обліку можна розділити на вже готові прикладні застосування, платформи для конфігурації предметної області та розробки абсолютно нової системи, що задовольняє всім прикладним задачам. Cтрімко розвиваються системи конфігурування предметної області на кшталт <<1С: Підприємство>>, що характеризуються стрімким зростанням попиту, розвитком; подібні платформи можна <<глибоко>> конфігурувати, що дозволяє адаптувати їх у великій кількості сфер діяльності для різних задач. Але у них є і недоліки: необхідність наявності на підприємстві експертів із підтримки та розробки таких систем (зокрема, програміста, та в деяких випадках --- адміністратора), що спричиняє необхідність витрат на оплату їх послуг та, як наслідок --- збільшує вартість розробки та обслуговування такої системи. [1]

Готові прикладні застосування, у свою чергу, не завжди можуть забезпечити повну функціональність для вирішення прикладних завдань, не завжди доступні для придбання у фінансовому плані та часто не надають ліцензії на зміну вихідного коду. Тоді підприємство, у випадку економічної вигідності (вартість розробки та супроводу прийнятна для замовника), приймає рішення щодо розробки нової системи, що повністю задовольняє вимогам.

Серед існуючих аналогів розглянуто системи для автоматизації бухгалтерського обліку та документообігу в організаціях, зокрема із готовими конфігураціями для навчальних закладів, систему ведення електронних курсів, а також платформу для формування форм та звітів. За результатами функціонального аналізу жоден аналог не покриває всю необхідну функціональність. При цьому деякі дозволяють реалізувати функціональність, якої бракує, окремим програмним модулем та інтегрувати його з системою, однак і таке використання не є виправданим з урахуванням малої частки покритої функціональності, обмежених можливостей засобів інтеграції та вартості систем.

Метою створення системи є зменшення часу, що витрачається працівниками кафедри на пов'язану з курсами роботу шляхом автоматизації набору слухачів на курси, реєстрацій сплат та формування звітності, а також для надання майбутнім слухачам можливості ознайомитися з наявними курсами та залишати заявки на реєстрацію за допомогою зовнішніх програмних сервісів.

Завданням роботи є розробка програмного продукту для підвищення якості роботи персоналу, що зумовлено реалізацією певних функціональних можливостей системи, таких як керування курсами, слухачами, викладачами, оплатами за курси, формування звітів, обробка заявок.

Створення програмного забезпечення супроводжується процесом розробки. Існує кілька моделей процесу розробки, кожна з яких описує свій підхід у вигляді задач чи діяльності, що має місце в ході процесу. Основними етапами, з яких складається сам процес, є: бізнес-моделювання, аналіз вимог, планування, розробка архітектури, кодування, тестування, налагодження, документування, впровадження та супровід.

В першому розділі <<Визначення бізнес-вимог>> дипломної роботи приведені вимоги бізнес-рівня, функціональні, не функціональні, середовища функціонування, кваліфікація користувачів, проведений аналіз існуючих аналогів, з виділенням  переваг і недоліків.

В другому розділі <<Проектування програмного продукту>> описано проектування архітектури системи, структури та організації концептуальних класів та бази даних.

В третьому розділі <<Конструювання програмного продукту>> представлені набір інструментальних засобів розробки та алгоритм програми, тестування функціональності системи та приклад її використання.

В четвертому розділі <<Розгортання програмного продукту>> наведено інструкції з встановлення та використання системи.
\end{sloppy}
