\bigbreak
\subsection{Модульне тестування}
\bigbreak
Перед початком тестування слід заповнити порожню базу даних тестовими даними (табл. 3.1--3.7). Тестування алгоритму розрахунку вартості курсу проведено методом <<білої скрині>> (таблиця 3.8), алгоритму аутентифікації та авторизації --- методом <<чорної скрині>> (таблиця 3.9).
\def\testhead{\hline Атрибут & \multicolumn{2}{c|}{Значення} \\\hline}
\tablefirsthead{\testhead}
\tablehead{\multicolumn{3}{l}{Продовження таблиці \thesection .\thetabcount} \\\hline\testhead}
\tablabel{Тестові дані для таблиці courses}
\begin{center}
\begin{supertabular}{|l|c|c|}
idCourse & 1 & 2 \\
Name & a & b \\
Description & a & b \\
isIndividual & 0 & 0 \\
idTeacher & 1 & 1 \\
idTeacher2 & 2 & 2 \\
Price & 15 & 0 \\
state & 0 & 0 \\
affectedBy & user1 & user2 \\
\hline
\end{supertabular}
\end{center}

\tablabel{Тестові дані для таблиці Course\_Listeners}
\begin{center}
\begin{tabular}{|l|c|c|c|c|}
\hline
Атрибут & \multicolumn{4}{c|}{Значення} \\
\hline
idCL & 1 & 2 & 3 & 4 \\
idCourse & 1 & 1 & 1 & 2 \\
idListener & 1 & 2 & 3 & 1 \\
mark & 0 & 0 & 0 & 0 \\
affectedBy & user1 & user1 & user1 & user1 \\
\hline
\end{tabular}
\end{center}

\tablabel{Тестові дані для таблиці teachers}
\begin{center}
\begin{tabular}{|l|c|c|}
\hline
Атрибут & \multicolumn{2}{c|}{Значення} \\
\hline
idTeacher & 1 & 2 \\
Name & a & b \\
Surname & a & b \\
Patronymic & a & b \\
Phone & 289 & 2893 \\
Email & b@a.b & c@d.c \\
degree & 1 & 2 \\
affectedBy & user1 & user2 \\
\hline
\end{tabular}
\end{center}

\newpage
\tablabel{Тестові дані для таблиці prices}
\begin{center}
\begin{tabular}{|l|c|c|c|}
\hline
Атрибут & \multicolumn{3}{c|}{Значення} \\
\hline
degree & 1 & 2 & 3 \\
deglab & Б. с. & Доц. к. т. н. & Проф. \\
salary & 15.62 & 23.24 & 35.38 \\
\hline
\end{tabular}
\end{center}

\tablabel{Тестові дані для таблиці lessons}
\begin{center}
\begin{tabular}{|l|c|c|c|c|c|c|}
\hline
Атрибут & \multicolumn{6}{c|}{Значення} \\
\hline
idLesson & 1 & 2 & 3 & 4 & 5 & 6 \\
idCourse & 1 & 1 & 1 & 2 & 2 & 2 \\
date & 1.4.15 & 2.4.15 & 3.4.15 & 4.4.15 & 5.4.15 & 6.4.15 \\
time & 13:00 & 13:00 & 13:00 & 13:00 & 13:00 & 13:00 \\
type & 1 & 1 & 2 & 1 & 2 & 2 \\
\hline
\end{tabular}
\end{center}

\tablabel{Тестові дані для таблиці users}
\begin{center}
\begin{tabular}{|l|c|c|c|}
\hline
Атрибут & \multicolumn{3}{c|}{Значення} \\
\hline
login & adm1 & user1 & user2 \\
password & $hash('asdf')$ & $hash('qwer')$ & $hash('zxcv')$ \\
salt & $<hash1>$ & $<hash2>$ & $<hash3>$ \\
type & 0 & 1 & 2 \\
sessionid & $<key1>$ & NULL & q \\
\hline
\end{tabular}
\end{center}

{\setlength{\tabcolsep}{0pt}
\tablabel{Тестові дані для таблиці coefficients}
\begin{center}
\begin{tabular}{|l|c|c|c|}
\hline
Атрибут & \multicolumn{3}{c|}{Значення} \\
\hline
name & bonus & others & personal \\
label & \parbox{4.5cm}{Нарахування на заробітну плату} & Інші послуги та утримки & Зарплата персоналу \\
value & 1.12 & 1.23 & 1.05 \\
\hline
\end{tabular}
\end{center}}

\tablabel{Тестові випадки для алгоритму розрахунку вартості курсу}
\newcounter{tcnt}\def\tcn{\addtocounter{tcnt}{1} \thetcnt}
\begin{tabular}{|c|c|c|c|c|}
\hline
\multirow{2}{23mm}{\centering \textbf{Test Case №}} &
\multicolumn{2}{c|}{\textbf{Вхідні дані}} &
\multirow{2}{27mm}{\centering \textbf{Очікуваний результат}} &
\multirow{2}{91mm}{\centering \textbf{Результат тестування (успішний (passed) / неуспішний (failed))}} \tabularnewline
\cline{2-3}
&id&full&&\tabularnewline
\hline
\tcn& 1	& true	& 60	& Passed \\
\tcn& 2	& true	& 15	& Passed \\
\tcn& 1	& false	& 20	& Passed \\
\tcn& 2	& false	& 15	& Passed \\
\hline
\end{tabular}

\setcounter{tcnt}{0}
\tablabel{Тестові випадки для алгоритму аутентифікації та авторизації}\\
\noindent\begin{tabular}{|c|c|c|c|c|}
\hline
\multirow{2}{18mm}{\centering \textbf{Test Case №}} &
\multicolumn{2}{c|}{\textbf{Вхідні дані}} &
\multirow{2}{27mm}{\centering \textbf{Очікуваний результат}} &
\multirow{2}{90mm}{\centering \textbf{Результат тестування (успішний (passed) / неуспішний (failed))}} \tabularnewline
\cline{2-3}
&login&pass&&\tabularnewline
\hline
\tcn& 'adm1'	& 'asdf'	& Вхід		& Passed \\
\tcn& 'adm1'	& 'qwer'	& Відмова	& Passed \\
\tcn& 'adm1'	& NULL		& Відмова	& Passed \\
\tcn& 'user3'	& 'qwer'	& Відмова	& Passed \\
\tcn& 'user3'	& 'sdfj'	& Відмова	& Passed \\
\tcn& 'user2'	& 'zxcv'	& Відмова	& Passed \\
\hline
\end{tabular}
